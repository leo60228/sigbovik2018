\documentclass{article}
    \usepackage[utf8]{inputenc}
    \usepackage{nopageno}
    \usepackage{hyperref}

\title{Optimizing the SIGBOVIK 2018 speedrun}
\author{leo60228  \\
        @leo60228}
\date{April 1, 2020}

\begin{document}
\maketitle
\begin{abstract}
SIGBOVIK 2018 \cite{sigbovik} is a popular game. However, very little work has been put into optimizing its speedrun. In this paper, I will demonstrate usage of pathfinding algorithms for the goal of making the optimal speedrun of SIGBOVIK.
\end{abstract}

\section{Motivation}
I was bored.

\section{Method}
I first looked at implementations of pathfinding algorithms in Rust. I initially wanted to use A* because it was popular. However, the API looked complicated. I found an implementation of DFS to use instead. This DFS implementation required me to define a start point, one or more end points, and a function to map from point to points. I implemented all three of these, using an enum of Start, Success, and Page, because I forgot that there was only one success page.

I used a depth first search algorithm in order to create the mapping. To do this, I clicked as far as possible along the first choices, adding to my mapping (a \texttt{match} block) as I went, then backtracking once I reached a cycle or the end. I then manually removed cycles because I wasn't sure if the DFS implementation I was using supported them and I didn't feel like writing my own.

After I had defined these three functions, I simply used the \texttt{pathfinding} \cite{pathfinding} crate's DFS function. This gave me an optimal path to program into LiveSplit One \cite{livesplit}, a tool for speedrunning that I've seen speedrunners use, so I probably should too if I'm speedrunning. I opened it on my phone and added the order that my code gave me.

I recorded my phone and computer screen while I began the speedrun. I then combined the phone and computer screen and attempted to sync them from memory in a video editor. I then uploaded it to YouTube \cite{youtube}.

\section{Results}
My speedrun was 17.88 seconds, which I'm pretty sure is a world record among the SIGBOVIK 2018 speedrunning community. It can be viewed at \url{https://youtu.be/VPrT8Y-aRRs}. My code will be available on GitHub at \url{https://github.com/leo60228/sigbovik2018} if I remember to make it non-private after SIGBOVIK.

\section{Appendix: Code}
\begin{verbatim}
use pathfinding::directed::dfs::dfs;

#[derive(Debug, PartialEq, Eq, Copy, Clone)]
pub enum Page {
    Start,
    Success,
    Page(isize),
}

macro_rules! pagevec {
    ($($page:expr),*) => {
        vec![$(Page::Page($page)),*]
    }
}

impl Page {
    pub fn successors(&self) -> Vec<Self> {
        match self {
            Page::Start => pagevec![47, 177, 205],
            Page::Page(47) => pagevec![153, 206],
            Page::Page(153) => pagevec![17],
            Page::Page(17) => pagevec![35, 135],
            Page::Page(35) => pagevec![51, 68],
            Page::Page(51) => pagevec![68], // cycle
            Page::Page(68) => vec![], // cycle
            Page::Page(135) => pagevec![124, 183],
            Page::Page(124) => pagevec![116, 88],
            Page::Page(116) => pagevec![208],
            Page::Page(208) => vec![Page::Success],
            Page::Page(88) => pagevec![208],
            Page::Page(183) => pagevec![207, 208],
            Page::Page(207) => vec![],
            Page::Page(206) => vec![],
            Page::Page(177) => pagevec![130, 117],
            Page::Page(130) => pagevec![154, 39],
            Page::Page(154) => pagevec![17],
            Page::Page(39) => vec![],
            Page::Page(117) => pagevec![87, 50, 28],
            Page::Page(87) => pagevec![69, 40, 28],
            Page::Page(69) => vec![], // cycle
            Page::Page(28) => vec![], // cycle
            Page::Page(50) => vec![], // cycle
            Page::Page(40) => pagevec![178], // cycle
            Page::Page(178) => pagevec![208], // cycle
            Page::Page(205) => vec![],
            Page::Success => pagevec![],
            _ => unimplemented!("{:?}", self),
        }
    }
}

fn main() {
    println!("{:?}", dfs(Page::Start, Page::successors, |x| x == &Page::Success));
}
\end{verbatim}

\section{Appendix 2}
I just realized I probably should have used more sections, so here's one.

\begin{thebibliography}{9}
    \bibitem{sigbovik}
        Association for Computational Heresy. Message from the Organizing Commitee. In \emph{Proceedings of SIGBOVIK 2018}. ACH, Pittsburgh, PA, USA, March 29, 2018.
    \bibitem{pathfinding}
        Samuel Tardieu. \texttt{pathfinding}. \url{https://crates.io/crates/pathfinding}
    \bibitem{livesplit}
        LiveSplit. LiveSplit One. \url{https://one.livesplit.org}
    \bibitem{youtube}
        Google. YouTube. \url{https://youtube.com}
\end{thebibliography}

\end{document}
